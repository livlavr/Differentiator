\documentclass[12pt]{article}
\usepackage[utf8]{inputenc}
\usepackage[russian]{babel}
\usepackage{xcolor}
\usepackage{makecell}
\usepackage{amsmath}
\usepackage{amssymb}
\usepackage{cancel}
\usepackage{listings}
\usepackage{tikz}
\linespread{1.5}
\definecolor{codegreen}{rgb}{0,0.6,0}
\definecolor{codegray}{rgb}{0.5,0.5,0.5}
\definecolor{codepurple}{rgb}{0.58,0,0.82}
\definecolor{backcolour}{rgb}{0.95,0.95,0.92}
\usepackage[
    ignoreheadfoot,
    top=2 cm,
    bottom=2 cm,
    left=2 cm,
    right=2 cm,
    footskip=1.0 cm,
]{geometry}
\begin{document}
\begingroup
    \centering
    \LARGE Дифференцирование\\
    \large \today \\[0.5em]
    \large Лаврущев Иван Б05-431\par
\endgroup
\tableofcontents
\section{Хмм, как бы здесь изъебнуться}
$$3 \cdot \cos(20 \cdot x)^{2} \cdot 20 \cdot \sin(20^{x})$$\\
\textbf{После упрощения получаем:}\\
$$3 \cdot \cos(20 \cdot x)^{2} \cdot 20 \cdot \sin(20^{x})$$\\
\section{ПОГНАЛИ}
\subsection{Задачка для советских яслей}
$$(3)' = 0$$
\subsection{Кто не посчитает получит пиздюлей}
$$(20)' = 0$$
\subsection{Пять утра я сижу блять фразы для дифференциатора пишу}
$$(x)' = 1$$
\subsection{Коллеги не падайте со стульев, тут полный шок}
$$(20 \cdot x)' = 0 \cdot x + 20 \cdot 1$$
\subsection{Какое милое говно}
$$(\cos(20 \cdot x))' = -1 \cdot \sin(20 \cdot x) \cdot 0 \cdot x + 20 \cdot 1$$
\subsection{Блин жрать охота, может пойти дошик заварить}
$$(\cos(20 \cdot x)^{2})' = 2 \cdot \cos(20 \cdot x)^{2 - 1} \cdot -1 \cdot \sin(20 \cdot x) \cdot 0 \cdot x + 20 \cdot 1$$
\subsection{Задачка для советских яслей}
$$(20)' = 0$$
\subsection{Не ну что это за рукоблудие}
$$(x)' = 1$$
\subsection{Хватит сидеть в тик токе, лучше уясни этот момент, коллега}
$$(20^{x})' = 20^{x} \cdot \ln(20) \cdot 1$$
\subsection{Кто посчитает, тот получит энергосик}
$$(\sin(20^{x}))' = \cos(20^{x}) \cdot 20^{x} \cdot \ln(20) \cdot 1$$
\subsection{Продавец пятерочки запросто продифференцирует данное выражение}
$$(20 \cdot \sin(20^{x}))' = 0 \cdot \sin(20^{x}) + 20 \cdot \cos(20^{x}) \cdot 20^{x} \cdot \ln(20) \cdot 1$$
\subsection{АААААААААААААА заебло}
$$(\cos(20 \cdot x)^{2} \cdot 20 \cdot \sin(20^{x}))' = 2 \cdot \cos(20 \cdot x)^{2 - 1} \cdot -1 \cdot \sin(20 \cdot x) \cdot 0 \cdot x + 20 \cdot 1 \cdot 20 \cdot \sin(20^{x}) + \cos(20 \cdot x)^{2} \cdot 0 \cdot \sin(20^{x}) + 20 \cdot \cos(20^{x}) \cdot 20^{x} \cdot \ln(20) \cdot 1$$
\subsection{Что бы посчитать такое... великое}
$$(3 \cdot \cos(20 \cdot x)^{2} \cdot 20 \cdot \sin(20^{x}))' = 0 \cdot \cos(20 \cdot x)^{2} \cdot 20 \cdot \sin(20^{x}) + 3 \cdot 2 \cdot \cos(20 \cdot x)^{2 - 1} \cdot -1 \cdot \sin(20 \cdot x) \cdot 0 \cdot x + 20 \cdot 1 \cdot 20 \cdot \sin(20^{x}) + \cos(20 \cdot x)^{2} \cdot 0 \cdot \sin(20^{x}) + 20 \cdot \cos(20^{x}) \cdot 20^{x} \cdot \ln(20) \cdot 1$$
\section{Хмм, как бы здесь изъебнуться}
$$0 \cdot \cos(20 \cdot x)^{2} \cdot 20 \cdot \sin(20^{x}) + 3 \cdot 2 \cdot \cos(20 \cdot x)^{2 - 1} \cdot -1 \cdot \sin(20 \cdot x) \cdot 0 \cdot x + 20 \cdot 1 \cdot 20 \cdot \sin(20^{x}) + \cos(20 \cdot x)^{2} \cdot 0 \cdot \sin(20^{x}) + 20 \cdot \cos(20^{x}) \cdot 20^{x} \cdot \ln(20) \cdot 1$$\\
\textbf{После упрощения получаем:}\\
$$3 \cdot 2 \cdot \cos(20 \cdot x) \cdot -1 \cdot \sin(20 \cdot x) \cdot 20 \cdot 20 \cdot \sin(20^{x}) + \cos(20 \cdot x)^{2} \cdot 20 \cdot \cos(20^{x}) \cdot 20^{x} \cdot 3$$\\
\LARGE ДА УЖ, СТУДЕНТАМ ИЗ ВШЭ ТАКОЕ И НЕ СНИЛОСЬ
\end{document}
