\documentclass[12pt]{article}
\usepackage[utf8]{inputenc}
\usepackage[russian]{babel}
\usepackage{xcolor}
\usepackage{makecell}
\usepackage{amsmath}
\usepackage{amssymb}
\usepackage{cancel}
\usepackage{listings}
\usepackage{tikz}
\linespread{1.5}
\definecolor{codegreen}{rgb}{0,0.6,0}
\definecolor{codegray}{rgb}{0.5,0.5,0.5}
\definecolor{codepurple}{rgb}{0.58,0,0.82}
\definecolor{backcolour}{rgb}{0.95,0.95,0.92}
\usepackage[
    ignoreheadfoot,
    top=2 cm,
    bottom=2 cm,
    left=2 cm,
    right=2 cm,
    footskip=1.0 cm,
]{geometry}
\begin{document}
\begingroup
    \centering
    \LARGE Дифференцирование\\
    \large \today \\[0.5em]
    \large Лаврущев Иван Б05-431\par
\endgroup
\tableofcontents
\section{Хмм, как бы здесь изъебнуться}
$$5 \cdot \sin(10 \cdot x^{3}) + \cos(20 \cdot x - 1)^{3}$$\\
\textbf{После упрощения получаем:}\\
$$5 \cdot \sin(10 \cdot x^{3}) + \cos(20 \cdot x - 1)^{3}$$\\
\section{ПОГНАЛИ}
\subsection{Задачка для советских яслей}
$$(5)' = 0$$
\subsection{Кто не посчитает получит пиздюлей}
$$(10)' = 0$$
\subsection{Пять утра я сижу блять фразы для дифференциатора пишу}
$$(x)' = 1$$
\subsection{Коллеги не падайте со стульев, тут полный шок}
$$(x^{3})' = 3 \cdot x^{3 - 1} \cdot 1$$
\subsection{Какое милое говно}
$$(10 \cdot x^{3})' = 0 \cdot x^{3} + 10 \cdot 3 \cdot x^{3 - 1} \cdot 1$$
\subsection{Блин жрать охота, может пойти дошик заварить}
$$(\sin(10 \cdot x^{3}))' = \cos(10 \cdot x^{3}) \cdot 0 \cdot x^{3} + 10 \cdot 3 \cdot x^{3 - 1} \cdot 1$$
\subsection{Задачка для советских яслей}
$$(5 \cdot \sin(10 \cdot x^{3}))' = 0 \cdot \sin(10 \cdot x^{3}) + 5 \cdot \cos(10 \cdot x^{3}) \cdot 0 \cdot x^{3} + 10 \cdot 3 \cdot x^{3 - 1} \cdot 1$$
\subsection{Не ну что это за рукоблудие}
$$(20)' = 0$$
\subsection{Хватит сидеть в тик токе, лучше уясни этот момент, коллега}
$$(x)' = 1$$
\subsection{Кто посчитает, тот получит энергосик}
$$(20 \cdot x)' = 0 \cdot x + 20 \cdot 1$$
\subsection{Продавец пятерочки запросто продифференцирует данное выражение}
$$(1)' = 0$$
\subsection{АААААААААААААА заебло}
$$(20 \cdot x - 1)' = 0 \cdot x + 20 \cdot 1 - 0$$
\subsection{Что бы посчитать такое... великое}
$$(\cos(20 \cdot x - 1))' = -1 \cdot \sin(20 \cdot x - 1) \cdot 0 \cdot x + 20 \cdot 1 - 0$$
\subsection{Задачка для советских яслей}
$$(\cos(20 \cdot x - 1)^{3})' = 3 \cdot \cos(20 \cdot x - 1)^{3 - 1} \cdot -1 \cdot \sin(20 \cdot x - 1) \cdot 0 \cdot x + 20 \cdot 1 - 0$$
\subsection{Ты реально не можешь понять, что}
$$(5 \cdot \sin(10 \cdot x^{3}) + \cos(20 \cdot x - 1)^{3})' = 0 \cdot \sin(10 \cdot x^{3}) + 5 \cdot \cos(10 \cdot x^{3}) \cdot 0 \cdot x^{3} + 10 \cdot 3 \cdot x^{3 - 1} \cdot 1 + 3 \cdot \cos(20 \cdot x - 1)^{3 - 1} \cdot -1 \cdot \sin(20 \cdot x - 1) \cdot 0 \cdot x + 20 \cdot 1 - 0$$
\section{Маленько поколдуем}
$$0 \cdot \sin(10 \cdot x^{3}) + 5 \cdot \cos(10 \cdot x^{3}) \cdot 0 \cdot x^{3} + 10 \cdot 3 \cdot x^{3 - 1} \cdot 1 + 3 \cdot \cos(20 \cdot x - 1)^{3 - 1} \cdot -1 \cdot \sin(20 \cdot x - 1) \cdot 0 \cdot x + 20 \cdot 1 - 0$$\\
\textbf{После упрощения получаем:}\\
$$5 \cdot \cos(10 \cdot x^{3}) \cdot 10 \cdot 3 \cdot x^{2} + 3 \cdot \cos(20 \cdot x - 1)^{2} \cdot -1 \cdot \sin(20 \cdot x - 1) \cdot 20$$\\
\LARGE УРААААААА ПОБЕДА, производная просто опущена как дешёвка
\end{document}
