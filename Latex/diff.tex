\documentclass[12pt]{article}
\usepackage[utf8]{inputenc}
\usepackage[russian]{babel}
\usepackage{xcolor}
\usepackage{makecell}
\usepackage{amsmath}
\usepackage{amssymb}
\usepackage{cancel}
\usepackage{listings}
\usepackage{tikz}
\linespread{1.5}
\definecolor{codegreen}{rgb}{0,0.6,0}
\definecolor{codegray}{rgb}{0.5,0.5,0.5}
\definecolor{codepurple}{rgb}{0.58,0,0.82}
\definecolor{backcolour}{rgb}{0.95,0.95,0.92}
\usepackage[
    ignoreheadfoot,
    top=2 cm,
    bottom=2 cm,
    left=2 cm,
    right=2 cm,
    footskip=1.0 cm,
]{geometry}
\begin{document}
\begingroup
    \centering
    \LARGE Дифференцирование\\
    \large \today \\[0.5em]
    \large Лаврущев Иван Б05-431\par
\endgroup
\tableofcontents
\section{Хмм, как бы здесь изъебнуться}
$\sin(15 \cdot x^{3}) + \cos(20 \cdot x)^{3}$\\
\textbf{После упрощения получаем:}\\
$\sin(15 \cdot x^{3}) + \cos(20 \cdot x)^{3}$\\
\section{ПОГНАЛИ}
\subsection{А вот здесь нихуя не тривиально}
$(15)' = 0$
\subsection{Земля тебе пухом}
$(x)' = 1$
\subsection{Хватит сидеть в тик токе, лучше уясни этот момент, коллега}
$(x^{3})' = 3 \cdot x^{3 - 1} \cdot 1$
\subsection{Ты реально не можешь понять, что}
$(15 \cdot x^{3})' = 0 \cdot x^{3} + 15 \cdot 3 \cdot x^{3 - 1} \cdot 1$
\subsection{Коллеги не падайте со стульев, тут полный шок}
$(\sin(15 \cdot x^{3}))' = \cos(15 \cdot x^{3}) \cdot 0 \cdot x^{3} + 15 \cdot 3 \cdot x^{3 - 1} \cdot 1$
\subsection{Четыре часа ночи я сижу блять фразы для дифференциатора пишу}
$(20)' = 0$
\subsection{А вот здесь нихуя не тривиально}
$(x)' = 1$
\subsection{Какое милое говно}
$(20 \cdot x)' = 0 \cdot x + 20 \cdot 1$
\subsection{Кто посчитает, тот получит энергосик}
$(\cos(20 \cdot x))' = -1 \cdot \sin(20 \cdot x) \cdot 0 \cdot x + 20 \cdot 1$
\subsection{Земля тебе пухом}
$(\cos(20 \cdot x)^{3})' = 3 \cdot \cos(20 \cdot x)^{3 - 1} \cdot -1 \cdot \sin(20 \cdot x) \cdot 0 \cdot x + 20 \cdot 1$
\subsection{Наши папы не работают в банках, и мы не катаемся на иномарках}
$(\sin(15 \cdot x^{3}) + \cos(20 \cdot x)^{3})' = \cos(15 \cdot x^{3}) \cdot 0 \cdot x^{3} + 15 \cdot 3 \cdot x^{3 - 1} \cdot 1 + 3 \cdot \cos(20 \cdot x)^{3 - 1} \cdot -1 \cdot \sin(20 \cdot x) \cdot 0 \cdot x + 20 \cdot 1$
\section{Посмотрим, что с этим можно сделать}
$\cos(15 \cdot x^{3}) \cdot 0 \cdot x^{3} + 15 \cdot 3 \cdot x^{3 - 1} \cdot 1 + 3 \cdot \cos(20 \cdot x)^{3 - 1} \cdot -1 \cdot \sin(20 \cdot x) \cdot 0 \cdot x + 20 \cdot 1$\\
\textbf{После упрощения получаем:}\\
$\cos(15 \cdot x^{3}) \cdot 15 \cdot 3 \cdot x^{2} + 3 \cdot \cos(20 \cdot x)^{2} \cdot -1 \cdot \sin(20 \cdot x) \cdot 20$\\
\LARGE ДА УЖ, СТУДЕНТАМ ИЗ ВШЭ ТАКОЕ И НЕ СНИЛОСЬ
\end{document}
