\documentclass[12pt]{article}
\usepackage{graphicx} % Required for inserting images
\usepackage[utf8]{inputenc}
\usepackage[russian]{babel}
\usepackage{xcolor}
\usepackage{makecell}
\usepackage{amsmath}
\usepackage{amssymb}
\usepackage{cancel}
\usepackage{listings}
\usepackage{tikz}
\linespread{1.5}
\definecolor{codegreen}{rgb}{0,0.6,0}
\definecolor{codegray}{rgb}{0.5,0.5,0.5}
\definecolor{codepurple}{rgb}{0.58,0,0.82}
\definecolor{backcolour}{rgb}{0.95,0.95,0.92}
\usepackage[
    ignoreheadfoot, % set margins without considering header and footer
    top=2 cm, % seperation between body and page edge from the top
    bottom=2 cm, % seperation between body and page edge from the bottom
    left=2 cm, % seperation between body and page edge from the left
    right=2 cm, % seperation between body and page edge from the right
    footskip=1.0 cm, % seperation between body and footer
    % showframe % for debugging
]{geometry} % for adjusting page geometry

\begin{document}

\begingroup
    \centering
    \LARGE Дифференцирование\\
    \large \today \\[0.5em]
    \large Лаврущев Иван Б05-431\par
\endgroup

\tableofcontents

\section{Упростим крокодила}
\LARGE $\cos(15 \cdot x^{3}) \cdot 45 \cdot x^{2} - 3 \cdot \cos(20 \cdot x)^{2} \cdot 20 \cdot \sin(20^{x})$
\normalsize
\subsection{Любому советскому школьнику очевидно, что}
$(15)' = 0$
\subsection{Студентов ПМИ до сих мучает вопрос какого хуя}
$(x)' = 1$
\subsection{Хотелось бы знать почему, но боюсь моя микроволновка не в силах осознать, что}
$(x^{3})' = 3 \cdot x^{3 - 1} \cdot 1$
\subsection{Заебало объяснять моим соседям по комнате, что}
$(15 \cdot x^{3})' = 0 \cdot x^{3} + 15 \cdot 3 \cdot x^{3 - 1} \cdot 1$
\subsection{Ну дальше совсем тривиальная задача, коллеги}
$(\cos(15 \cdot x^{3}))' = -1 \cdot \sin(15 \cdot x^{3}) \cdot 0 \cdot x^{3} + 15 \cdot 3 \cdot x^{3 - 1} \cdot 1$
\subsection{А вот здесь нихуя не тривиально}
$(45)' = 0$
\subsection{Зачем мне знать, что}
$(x)' = 1$
\subsection{Где в жизни мне пригодится, знание о том, что}
$(x^{2})' = 2 \cdot x^{2 - 1} \cdot 1$
\subsection{Продавец пятерочки запросто продифференцирует данное выражение}
$(45 \cdot x^{2})' = 0 \cdot x^{2} + 45 \cdot 2 \cdot x^{2 - 1} \cdot 1$
\subsection{Кис-кис кис-кис я котик ты котик}
$(\cos(15 \cdot x^{3}) \cdot 45 \cdot x^{2})' = -1 \cdot \sin(15 \cdot x^{3}) \cdot 0 \cdot x^{3} + 15 \cdot 3 \cdot x^{3 - 1} \cdot 1 \cdot 45 \cdot x^{2} + \cos(15 \cdot x^{3}) \cdot 0 \cdot x^{2} + 45 \cdot 2 \cdot x^{2 - 1} \cdot 1$
\subsection{АААААААААААААА заебло}
$(3)' = 0$
\subsection{Ты реально не можешь понять, что}
$(20)' = 0$
\subsection{Как тебя земля вообще носит, если ты не понимаешь, что}
$(x)' = 1$
\subsection{Ну такое выражение в детском саду учат дифференцировать}
$(20 \cdot x)' = 0 \cdot x + 20 \cdot 1$
\subsection{Хм, звучит отнюдь не тривиально}
$(\cos(20 \cdot x))' = -1 \cdot \sin(20 \cdot x) \cdot 0 \cdot x + 20 \cdot 1$
\subsection{Макс Тимошкин не смог взять эту производную}
$(\cos(20 \cdot x)^{2})' = 2 \cdot \cos(20 \cdot x)^{2 - 1} \cdot -1 \cdot \sin(20 \cdot x) \cdot 0 \cdot x + 20 \cdot 1$
\subsection{Хватит сидеть в тик токе, лучше уясни этот момент, коллега}
$(20)' = 0$
\subsection{Мама забери меня отсюда}
$(x)' = 1$
\subsection{В душе все мы немножко с шизой}
$(20^{x})' = 20^{x} \cdot \ln(20) \cdot 1$
\subsection{А мог бы сейчас в Бауманке чилить, но нет, захотел заниматься этой хуйней}
$(\sin(20^{x}))' = \cos(20^{x}) \cdot 20^{x} \cdot \ln(20) \cdot 1$
\subsection{Мяу}
$(20 \cdot \sin(20^{x}))' = 0 \cdot \sin(20^{x}) + 20 \cdot \cos(20^{x}) \cdot 20^{x} \cdot \ln(20) \cdot 1$
\subsection{Какое милое говно}
$(\cos(20 \cdot x)^{2} \cdot 20 \cdot \sin(20^{x}))' = 2 \cdot \cos(20 \cdot x)^{2 - 1} \cdot -1 \cdot \sin(20 \cdot x) \cdot 0 \cdot x + 20 \cdot 1 \cdot 20 \cdot \sin(20^{x}) + \cos(20 \cdot x)^{2} \cdot 0 \cdot \sin(20^{x}) + 20 \cdot \cos(20^{x}) \cdot 20^{x} \cdot \ln(20) \cdot 1$
\subsection{АААААААААААА что делать с этой залупой}
$(3 \cdot \cos(20 \cdot x)^{2} \cdot 20 \cdot \sin(20^{x}))' = 0 \cdot \cos(20 \cdot x)^{2} \cdot 20 \cdot \sin(20^{x}) + 3 \cdot 2 \cdot \cos(20 \cdot x)^{2 - 1} \cdot -1 \cdot \sin(20 \cdot x) \cdot 0 \cdot x + 20 \cdot 1 \cdot 20 \cdot \sin(20^{x}) + \cos(20 \cdot x)^{2} \cdot 0 \cdot \sin(20^{x}) + 20 \cdot \cos(20^{x}) \cdot 20^{x} \cdot \ln(20) \cdot 1$
\subsection{Земля тебе пухом}
\subsection{Наши папы не работают в банках, и мы не катаемся на иномарках}
\subsection{Нас не отправляют учиться в Лондон}
\subsection{Ничего на свете лучше неееету, чем взять производную вот ээээту}
\subsection{Задачка для советских яслей}
\subsection{Пять утра я сижу блять фразы для дифференциатора пишу}
\subsection{Дайте лучше мне пива, а не это говно}
\subsection{Блин жрать охота, может пойти дошик заварить}
\subsection{Что бы посчитать такое... великое}
\subsection{Коллеги не падайте со стульев, тут полный шок}
\subsection{Вы шокированы?}

\end{document}
